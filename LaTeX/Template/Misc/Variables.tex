% !TeX root = ../../main.tex

% Detalles de la tesis.
\title{Desarrollo de una aplicación contextual, apegada a la norma NOM-024-SSA3-2010} %-->Título de la tesis
\author{Daniel Humberto Ramírez Juárez}


%Director%
\advisor{ M. E. C. Rolando Pedro Gabriel. }
%Revisores%
%\coadvisorOne{Delightful Researcher}%
%\coadvisorTwo{Equally D. Researcher}%
%\committeeInternal{Person Inside}%

% Jurado%

%



% ... sobre el Grado
\degree{Licenciado en Informática}
\degreeyear{2019}
\degreemonth{Octubre}


% ... sobre la Escuela
\institute{Instituto de Informática}%
\department{Licenciatura en Informática}%
\university{Universidad de la Sierra Sur}%
\place{Miahuatlán de Porfirio Díaz, Oax., México.}%

% Definiciones
%\newtheorem{definicion}{Definición}
%\newtheorem{teorema}{Teorema}
%\newtheorem{corolario}{Corolario}
%\newtheorem{lema}{Lema}
%\newtheorem{proposicion}{Proposición}
%\newtheorem{problema}{Problema}
%\newtheorem{ejemplo}{Ejemplo}
%\newtheorem{Cuadro}{Tabla}
%\newtheorem{remark}{Observación}
%---------------------------------

%\spanishdecimal{.}
\DeclareGraphicsExtensions{.pdf}
\def\tablesp{\def\baselinestretch{1.1}\large\normalsize}
\def\subtema{\subsubsection*}


%\hypersetup{backref=true, colorlinks=true}
\spanishdecimal{.}
\DeclareGraphicsExtensions{.pdf}
\def\tablesp{\def\baselinestretch{1.1}\large\normalsize}
\def\subtema{\subsubsection*}
%----------------------------------
%\hyphenation{ abs-tracta ma-ni-pu-la-ci�n in-te-ra-cci�n }
     % ---- Separaciones de palabras
%\input{defmat}     % ---- Definiciones matem�ticas


%Algoritmo
%\definecolor{dkgreen}{rgb}{0,0.6,0}
%\definecolor{gray}{rgb}{0.5,0.5,0.5}
%\definecolor{mauve}{rgb}{0.58,0,0.82}

%\lstset{frame=tb,
%	language=C++,
%	aboveskip=3mm,
%	belowskip=3mm,
%	numbers=left,
%	showstringspaces=false,
%	columns=flexible,
%	basicstyle={\small\ttfamily},
%	numberstyle=\tiny\color{gray},
%	keywordstyle=\color{blue},
%	commentstyle=\color{dkgreen},
%	stringstyle=\color{mauve},
%	breaklines=true,
%	frame=leftline,
%	breakatwhitespace=true
%	tabsize=3
%}