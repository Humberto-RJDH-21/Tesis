\chapter{Conclusiones y trabajos futuros}
\label{Ch:Conclusiones}

%Del trabajo realizado en esta tesis se concluye que el PID
%\emph{wavelet} presenta mejores resultados comparados con los
%arrojados por el PID tradicional, esto para plantas de naturaleza
%estable. Dicha mejora se debe a la descomposici�n que el dise�ador
%puede realizar, en teor�a la descomposici�n puede ser infinita,
%pero en la pr�ctica es recomendable que sea finita, debido a las
%limitaciones, adem�s que entre m�s grande es la descomposici�n
%mayor es el n�mero de ganancias a ser sintonizadas. Para sistemas
%que son de naturaleza inestable el PID \emph{wavelet} no presenta
%buenos resultados, es por ello que se propone el PID
%\emph{wavenet}, el cual ser� tratado en el Cap�tulo
%\ref{Ch:ContPIDWavenet}.

\section{Conclusiones.}

En este trabajo de investigaci�n se dise�aron dos variantes del
controlador PID que hacen uso de la teor�a \emph{wavelet}. El
primero es el PIDMR que hace uso de una descomposici�n de la se�al
del error, donde dicha descomposici�n se efect�a mediante la
t�cnica multiresoluci�n y decodificaci�n sub-banda, este
controlador se emplea para controlar un motor de CD presentando
buenos resultados tanto en simulaci�n como en laboratorio, el
segundo controlador llamado \emph{wavenet} hace uso de redes
neuronales de base radial con funciones de activaci�n que son
\emph{wavelets} hijas, aqu� las redes se emplean para dos
prop�sitos (ver Figura \ref{Fig:ControlWavenet}), una para la
identificaci�n del sistema y otra para la sintonizaci�n de los
par�metros del PID \emph{wavenet}, este controlador tambi�n se
aplic� a un motor de CD obteni�ndose resultados satisfactorios, de
lo anterior se concluye que el objetivo de la tesis se cumpli�
satisfactoriamente.

\section{Trabajos futuros.}

Algunos de los trabajos futuros que se desprenden de esta tesis
son:

\begin{itemize}
    \item Encontrar un m�todo de sintonizaci�n del PIDMR.
    \item Hacer un an�lisis riguroso de la convergencia de los
    algoritmos.
    \item Hacer un estudio sobre la estabilidad en lazo cerrado
    para ambos esquemas propuestos en la tesis.
\end{itemize}
