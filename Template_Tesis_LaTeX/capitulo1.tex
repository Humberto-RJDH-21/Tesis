\chapter{Introducci�n}

Iniciar con la escritura de la tesis...

\clearpage


\section{Antecedentes}

Colocar los antecedentes, resultado del trabajo previo de investigaci�n... % colocaci�n de una secci�n dentro de un cap�tulo

\subsection{Subsecci�n} \label{haptica} %  \label{X} es una etiqueta de referencia cruzada

Ejemplo de una subsecci�n dentro del cap�tulo... y el uso de label \{ \} y ref \{ \} para referencias cruzadas


\section{nueva secci�n}\label{IH}

otro ejemplo de secci�n dentro del cap�tulo, como se observa en \ref{haptica}... %ejemplo de referencia cruzada con \label{} y \ref{}

Llamado a la bibliografia, archivo biliografia.bib, \cite{Abowd},\cite{C2link1}, entre otros...

\newpage


